\documentclass[11pt, tikz, multi=page]{standalone}
%\usepackage{sansmathfonts}		% Sans Serif equations
\usepackage[T1]{fontenc}			% Output font encoding for international characters
\usepackage[utf8]{inputenc}			% Encoding of files: utf8
%\renewcommand*\familydefault{\sfdefault} 	% Sans Serif as default font
\usepackage[spanish, es-tabla, es-nodecimaldot]{babel}
\usepackage{amssymb,amsmath}
\usepackage{bm}				% Bold symbols
\usepackage[only,Yup]{stmaryrd} % El símbolo de la estrella
\usepackage{overpic}
\usepackage{circuitikz}
\usetikzlibrary{positioning,shapes.misc}
\usetikzlibrary{babel}
% To use matlab2tikz
\usetikzlibrary{shapes}
\usepackage{pgfplots}
\usepackage{grffile}
\pgfplotsset{compat=newest}
\usetikzlibrary{plotmarks}
\usetikzlibrary{arrows.meta}
\usepgfplotslibrary{patchplots}

\newcommand{\path}[1] {lowres/#1}  

    \ctikzset{bipoles/resistor/height=0.15}
    \ctikzset{bipoles/resistor/width=0.4}
    \ctikzset{bipoles/diode/height=0.2}
    \ctikzset{bipoles/diode/width=0.2}
    \ctikzset{bipoles/capacitor/height=0.3}
    \ctikzset{bipoles/capacitor/width=0.1}



\definecolor{myBlueTHD}{rgb}{0.20810,0.16630,0.52920}%
\definecolor{myBlue}{rgb}{0.00000,0.44700,0.74100}%
\definecolor{myOrange}{rgb}{0.85000,0.32500,0.09800}%
\pgfplotsset{
	compat=1.16,
	myTHD/.style={
		width=9cm, height=4cm,
		scale only axis,
		bar shift auto,
		ylabel style={font=\footnotesize},
		axis background/.style={fill=white},
		title style={font=\bfseries\footnotesize},
		xmajorgrids, ymajorgrids,
		xticklabel style={/pgf/number format/fixed,/pgf/number format/precision=5},
		scaled x ticks=false,
		tick label style={font=\footnotesize}
	}
}

\newcommand{\csiblock}[3]
{ 
	node[block, minimum width=1.5cm, minimum height=1.5cm, #2, label=above:#3](#1){}
	(#1) node[xshift=0.05cm]{
		\begin{tikzpicture}
			\draw[line width=0.25mm] 
			(0,0.6cm) --++(0cm,-0.3cm) 
			(0,0.1cm) --++(0.15cm, 0.2cm) --++(-0.15cm, -0.2cm) --++(0,-0.2cm) 
			(0,-0.3cm) --++(0.15cm, 0.2cm) --++(-0.15cm, -0.2cm) --++(0,-0.3cm)
			(-0.43cm,0.6cm) --++(0cm,-0.3cm) 
			(-0.43cm,0.1cm) --++(0.15cm, 0.2cm) --++(-0.15cm, -0.2cm) --++(0,-0.2cm) 
			(-0.43cm,-0.3cm) --++(0.15cm, 0.2cm) --++(-0.15cm, -0.2cm) --++(0,-0.3cm)
			(0.43cm,0.6cm) --++(0cm,-0.3cm) 
			(0.43cm,0.1cm) --++(0.15cm, 0.2cm) --++(-0.15cm, -0.2cm) --++(0,-0.2cm) 
			(0.43cm,-0.3cm) --++(0.15cm, 0.2cm) --++(-0.15cm, -0.2cm) --++(0,-0.3cm);
		\end{tikzpicture}
	}
}

\newcommand{\mult}[1]
{ 
	node[draw, circle, fill={rgb,255:red,209; green,225; blue,243}, line width=0.25mm, minimum size=0.7cm](#1){}
	(#1) node{
		\begin{tikzpicture}
			\draw[line width=0.25mm] 
			(-0.25cm,0.25cm) -- (0.25cm,-0.25cm) 
			(-0.25cm,-0.25cm) -- (0.25cm,0.25cm); 
		\end{tikzpicture}
	}
}

\newcommand{\addersquare}[3]{ 
	node[draw, align=center, fill=gray!20, line width=0.25mm, minimum height=1.2cm, minimum width=1.2cm, inner sep=0pt](#1){}
	(#1) node{
		\begin{tikzpicture}
			\draw[line width=0.25mm] 
			(-0.35cm,0.3cm) node{#2}
			(-0.35cm,-0.3cm) node{#3}
			(0.35cm, 0cm) node{\phantom{$+$}};
		\end{tikzpicture}
	}
}

\newcommand{\lowpass}[2]{ 
	node[draw, align=center, fill=black!20, line width=0.25mm, minimum height=1cm, minimum width=1.3cm, inner sep=0pt, #2](#1){}
	(#1) node{
		\begin{tikzpicture}
			\draw[line width=0.25mm] 
			(-0.45cm,0.32cm) -- (0.25cm,0.32cm) -- (0.45cm, -0.32cm);
		\end{tikzpicture}
	}
}
