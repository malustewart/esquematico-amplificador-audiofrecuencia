\begin{tikzpicture}
\begin{axis}[
	xlabel = Re(loop gain),
	ylabel =  Im(loop gain),
	grid = both,
	width = \textwidth,
	]
	\addplot[black,line width=3pt] table [x index=1, y index = 2, col sep=comma] {data/falsodarlington_loopgain_cart_5.csv};
\end{axis}
\end{tikzpicture}
